%!TEX program = xelatex

% TODO: 让 \endnote 的标记使用黑体数字
% TODO: 让 \endnote 命令可以用于 \title 之中
% TODO: 让 \tdoc 与 \tpart 在目录中可以显示页码范围
% TODO: \nauthor 与 \neditor 之间的垂直间距可能需要微调

\documentclass[
  zihao=5,
  punct=kaiming,
  linespread=1.4,
  sub4section,
]{ctexbook}

% ------------------------------------------------------------------------------
% 导言区
% ------------------------------------------------------------------------------

\usepackage{bigfoot}
\usepackage{calc}
\usepackage{CJKfntef}
\usepackage{enotez}
\usepackage{etoolbox}
\usepackage{fancyhdr}
\usepackage{fontspec}
\usepackage{geometry}
\usepackage{graphicx}
\usepackage[hidelinks,hyperfootnotes=true]{hyperref}
\usepackage{lipsum}
\usepackage{perpage}
\usepackage{scrextend}
\usepackage{tikz}
\usepackage{titlesec}
\usepackage{titletoc}
\usepackage{truncate}
\usepackage{ulem}
\usepackage{xcolor}
\usepackage{xparse}
\usepackage{xunicode-addon}

\raggedbottom
\newsavebox{\headerbox}

\usepackage{xeCJKfntef}
\xeCJKsetup{underdot/symbol={\raisebox{-5pt}{◦}}}
\newcommand{\dotemph}[1]{\CJKunderdot{#1}}

% --------------------------------------
% 字体定义
% --------------------------------------

\setCJKmainfont{FZShuSong}
[
  Path           = fonts/,
  BoldFont       = FZHei,
  ItalicFont     = FZKai,
  BoldItalicFont = FZHei,
]
\setCJKsansfont{FZHei}[Path=fonts/]
\setCJKmonofont{HYFangSong}[Path=fonts/]

\setCJKfamilyfont{ss}{FZShuSong}[Path=fonts/]
\setCJKfamilyfont{fs}{FZFangSong}[Path=fonts/]
\setCJKfamilyfont{ht}{FZHei}[Path=fonts/]
\setCJKfamilyfont{kt}{FZKai}[Path=fonts/]
\setCJKfamilyfont{xb}{FZXiaoBiaoSong}[Path=fonts/]
\setCJKfamilyfont{xh}{FZXiHeiI}[Path=fonts/]
\setCJKfamilyfont{sz}{Berthold Baskerville Book}[Path=fonts/]

% 为封面设计的压缩字体
\newCJKfontfamily\coverxb{FZXiaoBiaoSong}[Path=fonts/, FakeStretch=0.85]
\newCJKfontfamily\coverxh{FZXiHeiI}[Path=fonts/, FakeStretch=0.85]

% 定义经过轻微压缩的字体族 (90%)
\newCJKfontfamily\modifiedss{FZShuSong}[Path=fonts/, FakeStretch=0.9]
\newCJKfontfamily\modifiedfs{FZFangSong}[Path=fonts/, FakeStretch=0.9]
\newCJKfontfamily\modifiedht{FZHei}[Path=fonts/, FakeStretch=0.9]
\newCJKfontfamily\modifiedkt{FZKai}[Path=fonts/, FakeStretch=0.9]
\newCJKfontfamily\modifiedxb{FZXiaoBiaoSong}[Path=fonts/, FakeStretch=0.9]
\newCJKfontfamily\modifiedxh{FZXiHeiI}[Path=fonts/, FakeStretch=0.9]

% 定义经过更轻微压缩的字体族 (95%)
\newCJKfontfamily\modifiedmodifiedss{FZShuSong}[Path=fonts/, FakeStretch=0.95]
\newCJKfontfamily\modifiedmodifiedfs{FZFangSong}[Path=fonts/, FakeStretch=0.95]
\newCJKfontfamily\modifiedmodifiedht{FZHei}[Path=fonts/, FakeStretch=0.95]
\newCJKfontfamily\modifiedmodifiedkt{FZKai}[Path=fonts/, FakeStretch=0.95]
\newCJKfontfamily\modifiedmodifiedxb{FZXiaoBiaoSong}[Path=fonts/,
FakeStretch=0.95]
\newCJKfontfamily\modifiedmodifiedxh{FZXiHeiI}[Path=fonts/, FakeStretch=0.95]

% 标点符号样式设置
\xeCJKEditPunctStyle{kaiming}
{
  fixed-margin-ratio = 0.000,
  mixed-margin-ratio = 1.000,
  middle-margin-ratio = 0.000,
}

% --------------------------------------
% 版面与边距
% --------------------------------------

\geometry{
  paperwidth  = 437bp,
  paperheight = 613bp,
  top         = 94.3bp,
  bottom      = 60.0bp,
  left        = 76.5bp,
  right       = 66.5bp,
  headsep     = 25.0bp,
  footskip    = 17.0bp,
  footnotesep = 11bp,
}

% --------------------------------------
% 页眉、页脚与页码
% --------------------------------------

\pagestyle{fancy}

\fancyhf{}

% 页眉与页脚

\renewcommand{\headrulewidth}{0bp}

\fancyhead[RO]{\rightmark}
\fancyhead[LE]{\leftmark}

\renewcommand{\chaptermark}[1]{%
  \sbox{\headerbox}{\small\modifiedxh#1}%
  \ifdim\wd\headerbox > 20em
  \def\theheadermark{\truncate[……]{20em}{\small\modifiedxh#1}}%
  \else
  \def\theheadermark{\small\modifiedxh#1}%
  \fi
  \markboth{\theheadermark}{}%
}

\renewcommand{\sectionmark}[1]{%
  \sbox{\headerbox}{\small\modifiedxh#1}%
  \ifdim\wd\headerbox > 20em
  \def\theheadermark{\truncate[……]{20em}{\small\modifiedxh#1}}%
  \else
  \def\theheadermark{\small\modifiedxh#1}%
  \fi
  \markright{\theheadermark}%
}

% 页码

\fancyfoot[RO]{\fontspec{Berthold Baskerville Book}[Path=fonts/]\thepage}
\fancyfoot[LE]{\fontspec{Berthold Baskerville Book}[Path=fonts/]\thepage}

\assignpagestyle{\part}{empty}
\fancypagestyle{empty}{%
  \fancyhf{}%
  \fancyfoot{}%
}

\fancypagestyle{plain}{%
  \fancyhf{}%
  \fancyfoot[RO]{%
    \makebox[0pt][l]{%
      \hspace{3.5bp}%
      \fontspec{Berthold Baskerville Book}[Path=fonts/]\thepage
    }%
  }%
  \fancyfoot[LE]{\fontspec{Berthold Baskerville Book}[Path=fonts/]\thepage}%
}

% --------------------------------------
%  命令导入
% --------------------------------------

% \AtBeginUTFCommand[\textcircled]{\begroup\EnclosedNumbers}
% \AtEndUTFCommand[\textcircled]{\endgroup}

\xeCJKDeclareCharClass{Default}{"24EA, "2460->"2473, "3251->"32BF}
\newfontfamily\EnclosedNumbers{FZShuSong}[Path=fonts/]

\makeatletter

\setlength{\skip\footins}{2\baselineskip plus 2\baselineskip minus 2\baselineskip}

% --------------------------------------
% nauthor
% --------------------------------------

\DeclareNewFootnote{B}

\newcommand\defaultfootnoterule{\vskip 8.0bp\hrule width 74bp height 0.3bp\vskip 8.0bp}

\let\fn@footnoteB\thefootnoteB
\let\fn@fnfootnoteB\footnoteB

\renewcommand{\thefootnoteB}{%
  \CJKfamily{fs}%
  \fontsize{8.0bp}{8.0bp}\selectfont%
  (\fn@footnoteB)%
}

\newcommand{\nauthor}[1]{%
  \deffootnote[5em]{0em}{0em}{%
    \raisebox{0.3bp}{%
      \thefootnotemark\hspace{0.5em}%
    }%
  }%
  \fn@fnfootnoteB{%
    \CJKfamily{fs}%
    \fontsize{9.0bp}{9.5bp}\selectfont%
    #1%
  }%
  \hspace{-0.2em}%
}

% --------------------------------------
% neditor
% --------------------------------------

\DeclareNewFootnote{C}

\renewcommand\extrafootnoterule{\vskip -4.0bp\hrule width \textwidth height 0.3bp\vskip 8.0bp}

\let\fn@fnfootnoteC\footnoteC

\renewcommand\thefootnoteC{%
  \EnclosedNumbers
  \textcircled{%
    \arabic{footnoteC}%
  }%
}

\newcommand{\neditor}[1]{%
  \hspace{0.02em}%
  \deffootnote{2em}{0em}{%
    \raisebox{0.4bp}{%
      \makebox[2em][l]{\thefootnotemark}%
    }%
  }%
  \fn@fnfootnoteC{%
    \CJKfamily{fs}%
    \fontsize{9.0bp}{9.5bp}\selectfont%
    #1%
  }%
}

\MakePerPage{footnoteC}

\makeatother

% --------------------------------------
% nend
% --------------------------------------

\setenotez{
  list-heading=
  \chapter*{\CJKfamily{kt}\fontsize{16.5bp}{16.5bp}\selectfont\ziju{1.500}{注释}},
  backref=true
}

\renewcommand\enmark[1]{%
  {\CJKfamily{ss}\fontsize{9.0bp}{9.75bp}\selectfont #1}\hspace{1em}%
}

\let\originalendnote\endnote

\newcommand{\nend}[1]{%
  \hspace{0.05em}%
  \originalendnote{%
    \begingroup
    \leftskip=2.9em\relax%
    \rightskip=0.1em\relax%
    \CJKfamily{ss}%
    \fontsize{9.0bp}{9.75bp}\selectfont%
    \ziju{0.050}%
    #1%
    \par%
    \endgroup%
    \vspace{-0.6em}%
  }%
  \hspace{-0.05em}%
}

\fancypagestyle{endnotestyle}{%
  \fancyhf{}%
  \renewcommand{\headrulewidth}{0bp}%

  \fancyhead[RO]{\small\CJKfamily{xh}\scalebox{0.9}[1.0]{\ziju{1.5pt}{注释}}}
  \fancyhead[LE]{\small\CJKfamily{xh}\scalebox{0.9}[1.0]{\ziju{1.5pt}{注释}}}

  \fancyfoot[RO]{%
    \makebox[0pt][l]{%
      \hspace{3.5bp}%
      \fontspec{Berthold Baskerville Book}[Path=fonts/]\thepage
    }%
  }
  \fancyfoot[LE]{\hspace{-4.5bp}\makebox[0pt][r]{\fontspec{Berthold Baskerville Book}[Path=fonts/]\thepage}}
}

\ctexset{
  tocdepth    = 10,
  secnumdepth = 6,
}

\renewcommand{\contentsname}{%
  \CJKfamily{xb}%
  \fontsize{16.5bp}{16.5bp}\selectfont\ziju{1.500}%
  目录%
}

% --------------------------------------
% intro
% --------------------------------------

\titleclass{\intro}{straight}[\part]

\newcounter{intro}[part]
\renewcommand\theintro{\hspace{-1em}}

\titlecontents{intro}[0bp]%
{%
  \CJKfamily{fs}%
  \fontsize{10.5bp}{10.5bp}\selectfont\ziju{0.050}%
  \vspace{3bp}%
}%
{}%
{}%
{\hspace{3bp}\titlerule*[.4em]{$\cdot$}\contentspage}

\newcommand\introbox{%
  \centering%
  \begin{minipage}{0.80\textwidth}%
    \modifiedmodifiedxb%
    \fontsize{16.5bp}{18.0bp}\selectfont\ziju{0.050}%
    \centering%
  }

  \titleformat{\intro}[display]%
  {\introbox}{}{0bp}%
  {}[%
\end{minipage}]

\titlespacing{\intro}%
{0bp}%
{0bp}%
{18.0bp}%

% --------------------------------------
% part
% --------------------------------------

\titlecontents{part}[0bp]%
{%
  \fontspec{Berthold Baskerville Book}[Path=fonts/]%
  \fontsize{14.5bp}{14.5bp}\selectfont\ziju{0.100}%
  \centering%
  \vspace{7.5bp}%
}%
{}%
{}%
{}[\vspace{5.5bp}]%

\newcommand\partbox{%
  \centering%
  \begin{minipage}{0.80\textwidth}%
    \fontspec{Berthold Baskerville Book}[Path=fonts/]%
    \fontsize{32.0bp}{32.0bp}\selectfont\ziju{0.100}%
    \centering%
  }

  \titleformat{\part}[display]%
  {\partbox}{}{0bp}%
  {}[%
\end{minipage}]

\ctexset{
  part / name   = ,
  part / number = \hspace{-1em},
}

\ctexset{
  part / fixskip    = true,
  part / afterskip  = ,
  part / beforeskip = ,
}

% --------------------------------------
% chapter
% --------------------------------------

\titlecontents{chapter}[0bp]%
{%
  \CJKfamily{xb}%
  \fontsize{9.5bp}{9.5bp}\selectfont\ziju{0.150}%
  \vspace{3bp}%
}%
{}%
{}%
{\hspace{3bp}\titlerule*[.4em]{$\cdot$}\contentspage}

\newcommand\chapterbox{%
  \centering%
  \begin{minipage}{0.80\textwidth}%
    \modifiedmodifiedxb%
    \fontsize{16.5bp}{18.0bp}\selectfont\ziju{0.050}%
    \centering%
  }

  \titleformat{\chapter}[display]%
  {\chapterbox}{}{0bp}%
  {}[%
\end{minipage}]

\titlespacing{\chapter}%
{0bp}%
{0bp}%
{18.0bp}%

\ctexset{
  chapter / fixskip    = true,
  chapter / afterskip  = ,
  chapter / beforeskip = ,
}

% --------------------------------------
% section
% --------------------------------------

\titlecontents{section}[2em]%
{%
  \CJKfamily{fs}%
  \fontsize{9.5bp}{9.5bp}\selectfont\ziju{-0.050}%
  \vspace{3bp}%
}%
{\contentspush{\thecontentslabel\hspace{1em}}}%
{}%
{\hspace{3bp}\titlerule*[.4em]{$\cdot$}\contentspage}

\newcommand\sectionbox{%
  \centering%
  \begin{minipage}{0.50\textwidth}%
    \modifiedmodifiedxh%
    \fontsize{14.5bp}{15.5bp}\selectfont\ziju{0.025}%
    \centering%
  }

  \titleformat{\section}[display]%
  {\sectionbox}%
  {\ziju{0.5}第\chinese{section}篇}%
  {5.0bp}%
  {}%
  [%
\end{minipage}]

\ctexset{
  section / name   = {第,篇},
  section / number = \chinese{section},
}

\ctexset{
  section / fixskip    = true,
  section / afterskip  = ,
  section / beforeskip = ,
}

\titlespacing{\section}%
{0bp}%
{28.0bp}%
{26.0bp}%

% --------------------------------------
% subsection
% --------------------------------------

\titlecontents{subsection}[2em]%
{%
  \CJKfamily{ss}%
  \fontsize{9.0bp}{9.0bp}\selectfont\ziju{0.010}%
  \vspace{3bp}%
}%
{\contentspush{\thecontentslabel\hspace{1em}}}%
{}%
{\hspace{3bp}\titlerule*[.4em]{$\cdot$}\contentspage}

\newcommand\subsectionbox{%
  \centering%
  \begin{minipage}{0.70\textwidth}%
    \modifiedmodifiedss%
    \fontsize{14.0bp}{15.0bp}\selectfont\ziju{0.025}%
    \centering%
  }

  \titleformat{\subsection}[display]%
  {\subsectionbox}%
  {\ziju{0.5}第\chinese{subsection}章}%
  {5.0bp}%
  {}%
  [%
\end{minipage}]

\ctexset{
  subsection / name   = {第,章},
  subsection / number = \chinese{subsection},
}

\ctexset{
  subsection / fixskip    = true,
  subsection / afterskip  = ,
  subsection / beforeskip = ,
}

\titlespacing{\subsection}%
{0bp}%
{14.5bp}%
{13.0bp}%

% --------------------------------------
% subsubsection
% --------------------------------------

\titlecontents{subsubsection}[4em]%
{%
  \CJKfamily{ss}%
  \fontsize{9.0bp}{9.0bp}\selectfont\ziju{0.010}%
  \vspace{3bp}%
}%
{\contentspush{\thecontentslabel\hspace{1em}}}%
{}%
{\hspace{3bp}\titlerule*[.4em]{$\cdot$}\contentspage}

\newcommand\subsubsectionbox[1]{%
  \begin{center}%
    \begin{minipage}{0.60\textwidth}%
      \modifiedmodifiedss%
      \fontsize{12.5bp}{14.5bp}\selectfont\ziju{0.025}%
      \centering%
      #1%
    \end{minipage}%
  \end{center}%
}

\ctexset{
  subsubsection / name       = {,.},
  subsubsection / number     = \arabic{subsubsection},
  subsubsection / format     = \subsubsectionbox,
  subsubsection / aftername  = ,
}

\ctexset{
  subsubsection / fixskip    = true,
  subsubsection / afterskip  = ,
  subsubsection / beforeskip = ,
}

\titlespacing{\subsubsection}%
{0bp}%
{19.5bp}%
{9.0bp}%

% --------------------------------------
% paragraph
% --------------------------------------

% \newlength{\paragraphhangindent}
% \newcommand{\calculateparagraphhangindent}{%
%   \settowidth{\paragraphhangindent}{%
%     \CJKfamily{ss}\fontsize{12.5bp}{12.5bp}\selectfont\ziju{0.010}%
%     (a)%
%   }%
% }
% \calculateparagraphhangindent

\titlecontents{paragraph}[6em]%
{%
  \CJKfamily{ss}%
  \fontsize{9.0bp}{9.0bp}\selectfont\ziju{0.010}%
  \vspace{3bp}%
}%
{\contentspush{\thecontentslabel\hspace{1em}}}%
{}%
{\hspace{3bp}\titlerule*[.4em]{$\cdot$}\contentspage}

% \newcommand\paragraphbox[1]{%
%   \begin{center}%
%     \begin{varwidth}{0.80\textwidth}%
%       \modifiedmodifiedss%
%       \fontsize{12.5bp}{12.5bp}\selectfont\ziju{0.010}%
%       %  FIXME: 以下两行无效
%       \hangindent=\paragraphhangindent%
%       \hangafter=1%
%       #1%
%     \end{varwidth}%
%   \end{center}%
% }

\newcommand\paragraphbox[1]{%
  \begin{center}%
    \begin{minipage}{0.80\textwidth}%
      \modifiedmodifiedss%
      \fontsize{12.5bp}{12.5bp}\selectfont\ziju{0.010}%
      \centering%
      #1%
    \end{minipage}%
  \end{center}%
}

\ctexset{
  paragraph / name       = {\CJKfamily{ss}(,\CJKfamily{ss})},
  paragraph / number     = \alph{paragraph},
  paragraph / format     = \paragraphbox,
  paragraph / aftername  = ,
}

\ctexset{
  paragraph / fixskip    = true,
  paragraph / afterskip  = ,
  paragraph / beforeskip = ,
}

\titlespacing{\paragraph}%
{0bp}%
{18.0bp}%
{9.0bp}%

% --------------------------------------
% subparagraph
% --------------------------------------

\titlecontents{subparagraph}[8em]%
{%
  \CJKfamily{ss}%
  \fontsize{9.0bp}{9.0bp}\selectfont\ziju{0.010}%
  \vspace{3bp}%
}%
{\contentspush{\thecontentslabel\hspace{1em}}}%
{}%
{\hspace{3bp}\titlerule*[.4em]{$\cdot$}\contentspage}

\newcommand\subparagraphbox[1]{%
  \begin{center}%
    \begin{minipage}{0.80\textwidth}%
      \CJKfamily{ht}%
      \fontsize{10.5bp}{10.5bp}\selectfont\ziju{0.010}%
      \centering%
      #1%
    \end{minipage}%
  \end{center}%
}

\ctexset{
  subparagraph / name         = {\CJKfamily{ss}(,\CJKfamily{ss})},
  subparagraph / number       = \arabic{subparagraph},
  subparagraph / format       = \subparagraphbox,
  subparagraph / numberformat = \bf,
  subparagraph / aftername    = ,
}

\ctexset{
  subparagraph / fixskip      = true,
  subparagraph / afterskip    = ,
  subparagraph / beforeskip   = ,
}

\titlespacing{\subparagraph}%
{0bp}%
{0bp}%
{-1bp}%

% ------------------------------------------------------------------------------
% \tintro                -> 前言     (可选)
% \tyear                 -> 年份标题 (可选)
% \tdoc                  -> 文献标题
% \tpart                 -> 篇       (可选)
% \tchapter(nonum)       -> 章
% \tsection(nonum)       -> 节   (如: 1. )
% \tsubsection(nonum)    -> 子节 (如: A. )
% \tsubsubsection(nonum) -> 小节 (如: (1))
% ------------------------------------------------------------------------------

% tintro

\newcommand{\tintro}[1]{%
  \cleardoublepage%
  \intro{#1}\label{#1}%
}

% tyear

\newcommand{\tyear}[1]{%
  \cleardoublepage%
  \thispagestyle{empty}%
  \part{#1}\label{#1}%
}

% tdoc

\newcommand{\tdoc}[1]{%
  \cleardoublepage%
  \thispagestyle{empty}%
  \let\oldthispagestyle\thispagestyle%
  \renewcommand{\thispagestyle}[1]{}%
  \chapter{#1}\label{#1}%
  \let\thispagestyle\oldthispagestyle%
}

% tpart

\newcommand{\tpart}[1]{%
  \cleardoublepage%
  \section{#1}\label{#1}%
}

\newcommand{\tpartnonum}[1]{%
  \cleardoublepage%
  \section*{#1}\label{#1}%
  \addcontentsline{toc}{section}{#1}%
  \partmark{#1}%
}

% tchapter

\newcommand{\tchapter}[1]{%
  \clearpage%
  \subsection{#1}\label{#1}%
}

\newcommand{\tchapternonum}[1]{%
  \clearpage%
  \subsection*{#1}\label{#1}%
  \addcontentsline{toc}{subsection}{#1}%
  \chaptermark{#1}%
}

% 为 chapter 添加副标题
\newcommand{\tvicechapter}[1]{%
  \vspace{-3.0bp}%
  \begin{center}%
    \begin{minipage}[t]{\textwidth}%
      \CJKfamily{fs}%
      \fontsize{10.5bp}{10.5bp}\selectfont\ziju{0.010}%
      \centering%
      #1
    \end{minipage}%
    \vspace{18.5bp}%
  \end{center}%
}

% tsection

\newcommand{\tsection}[1]{%
  \subsubsection{#1}\label{#1}%
}

\newcommand{\tsectionnonum}[1]{%
  \subsubsection*{#1}\label{#1}%
  \addcontentsline{to}{subsubsection}{#1}%
  \sectionmark{#1}%
}

% tsubsection

\newcommand{\tsubsection}[1]{%
  \paragraph{#1}\label{#1}%
}

\newcommand{\tsubsectionnonum}[1]{%
  \paragraph*{#1}\label{#1}%
  \addcontentsline{toc}{paragraph}{#1}%
}

% tsubsubsection

\newcommand{\tsubsubsection}[1]{%
  \subparagraph{#1}\label{#1}%
}

\newcommand{\tsubsubsectionnonum}[1]{%
  \subparagraph*{#1}\label{#1}%
  \addcontentsline{toc}{subparagraph}{#1}%
}

% important

\newcommand{\important}[1]{%
  \textbf{#1}%
}

% italic

\newcommand{\italic}[1]{%
  \textit{#1}%
}

% underdot

\newcommand{\underdot}[1]{%
  \CJKunderdot{#1}%
}

% hr
% 添加一条水平分割线

\newcommand{\hr}{%
  \vspace{0.3em}%
  {%
    \centering%
      *     *     *%
    \par%
  }%
}

% quo
% 添加一段引言

\renewenvironment{quote}
{%
  \list{}{\rightmargin=0pt \leftmargin=0pt}%
    \item[]\relax\vspace{0.2em}%
  \fontsize{9.0bp}{9.75bp}\selectfont%
  \hspace*{2em}\parindent=2em\parskip=0pt%
}%
{%
  \vspace{-0.15em}\endlist
}

\newcommand{\quo}[1]{
  \begin{quote}%
    #1%
  \end{quote}%
}

% closing
% 添加落款信息

\newcommand{\closing}[2]{%
  \vspace{10.5bp}%
  \begin{flushright}%
    \CJKfamily{fs}%
    \fontsize{9.0bp}{9.5bp}\selectfont\ziju{0.010}%
    #1
    \if\relax#2\relax%
    \else%
    \par#2%
    \fi%
  \end{flushright}%
  \vspace{10.5bp}%
}

% info
% 添加作者、出版及编辑等信息

\newcommand{\info}[4]{%
  \noindent
  \begin{minipage}[t]{0.4\textwidth}
    \begin{minipage}[t]{\textwidth}
      \fontsize{9.0bp}{9.5bp}\selectfont\ziju{0.010}#1%
    \end{minipage}

    \rule{0pt}{0.6em}

    \begin{minipage}[t]{\textwidth}
      \fontsize{9.0bp}{9.5bp}\selectfont\ziju{0.010}#2%
    \end{minipage}
  \end{minipage}%
  \hspace{0.2\textwidth}%
  \begin{minipage}[t]{0.4\textwidth}
    \begin{minipage}[t]{\textwidth}
      \fontsize{9.0bp}{9.5bp}\selectfont\ziju{0.010}#3%
    \end{minipage}

    \rule{0pt}{0.6em}

    \begin{minipage}[t]{\textwidth}
      \fontsize{9.0bp}{9.5bp}\selectfont\ziju{0.010}#4%
    \end{minipage}
  \end{minipage}
}

% TODO:为 \img 与 \tbl 添加(可选的):
% 1. 递增的编号
% 2. 随着章节递增的编号

% img
% 在当前位置插入一张图像

\NewDocumentCommand{\img}{O{1} m O{}}{%
  \par\vspace{\dimexpr\baselineskip - 1em\relax}%
  \noindent%
  \makebox[\textwidth][c]{%
    \includegraphics[width=#1\textwidth]{#2}%
  }%
  \IfValueT{#3}{%
    \par\vspace{-0.5em}%
    {\begin{center}
      \CJKfamily{ss}%
      \fontsize{9.0bp}{9.5bp}\selectfont
      \parbox{0.8\dimexpr#1\textwidth\relax}{\centering #3}
    \end{center}}%
  }%
  \par\vspace{\dimexpr\baselineskip - 1.8em\relax}%
}


% tbl
% 在当前位置插入一张表格(标题在上)

\NewDocumentCommand{\tbl}{O{1} m O{}}{%
  \par\vspace{\dimexpr\baselineskip - 1em\relax}%
  \IfValueT{#3}{%
    {\begin{center}
      \CJKfamily{ss}%
      \fontsize{9.0bp}{9.5bp}\selectfont
      \parbox{0.8\dimexpr#1\textwidth\relax}{\centering #3}
    \end{center}}%
    \par\vspace{-0.5em}%
  }%
  \noindent%
  \makebox[\textwidth][c]{%
    \includegraphics[width=#1\textwidth]{#2}%
  }%
  \par\vspace{\dimexpr\baselineskip - 1.8em\relax}%
}

% todo
% 在当前位置插入一个占位框

\newcommand{\todo}{%
  \par\vspace{12bp}%
  \noindent%
  \makebox[\textwidth][c]{%
    \begin{tikzpicture}%
      \node[%
        draw,%
        rectangle,%
        line width=1pt,%
        minimum height=5\baselineskip,%
        text width=0.8\textwidth,%
        align=center%
      ]%
      {这里应该添加一张图像或者一段文字};%
    \end{tikzpicture}%
  }%
  \par\vspace{4bp}%
}

\input{personal_cmd}

% ------------------------------------------------------------------------------
% 正文
% ------------------------------------------------------------------------------

\begin{document}

% --------------------------------------

% ------------------
% 封面
% ------------------

\cleardoublepage%
\thispagestyle{empty}%
{%
  \vspace*{\fill}%
  \begin{center}%
    \vspace{-12.5em}
    {\CJKfamily{coverxb}\fontsize{38bp}{47.5bp}\selectfont\ziju{0.200} 模版样本\par}
  \end{center}
  \vspace*{\fill}%
}
\cleardoublepage%

% ------------------
% 宣言
% ------------------

\cleardoublepage%
\thispagestyle{empty}%
\definecolor{deepred}{rgb}{0.6, 0, 0}
{\color{deepred}\CJKfamily{xb}\fontsize{14bp}{14.5bp}\selectfont\ziju{0.100}%
  \center 这是一段题词,可以是名言、格言或其他内容。
\par}

\frontmatter
\pagenumbering{arabic}


% --------------------------------------

\xeCJKsetup{CJKglue={\hskip 0bp plus 0.02\baselineskip minus
0.000\baselineskip}}
\setlength{\parskip}{0bp}

\clearpage
\fancyhf{}
\renewcommand{\headrulewidth}{0bp}

\fancyhead[RO]{\small\CJKfamily{xh}\scalebox{0.9}[1.0]{\ziju{1.5pt}{目录}}}
\fancyhead[LE]{\small\CJKfamily{xh}\scalebox{0.9}[1.0]{\ziju{1.5pt}{目录}}}
\fancyfoot[RO]{%
  \makebox[0pt][l]{%
    \hspace{3.5bp}%
    \fontspec{Berthold Baskerville Book}[Path=fonts/]\thepage
  }%
}
\fancyfoot[LE]{\hspace{-4.5bp}\makebox[0pt][r]{\fontspec{Berthold
Baskerville Book}[Path=fonts/]\thepage}}
\pagestyle{fancy}

\tableofcontents

\mainmatter

\clearpage
\fancyhf{}
\renewcommand{\headrulewidth}{0bp}

\fancyhead[RO]{\rightmark}
\fancyhead[LE]{\leftmark}
\fancyfoot[RO]{%
  \makebox[0pt][l]{%
    \hspace{3.5bp}%
    \fontspec{Berthold Baskerville Book}[Path=fonts/]\thepage
  }%
}
\fancyfoot[LE]{\hspace{-4.5bp}\makebox[0pt][r]{\fontspec{Berthold
Baskerville Book}[Path=fonts/]\thepage}}
\pagestyle{fancy}

\markboth{}{}

% --------------------------------------

\tintro{前言}

这是一段没有意义的话,用来模拟正文内容。这是一段没有意义的话,用来模拟正文内容。这是一段没有意义的话,用来模拟正文内容。这是一段没有意义的话,用来模拟正文内容。这是一段没有意义的话,用来模拟正文内容。这是一段没有意义的话,用来模拟正文内容。这是一段没有意义的话,用来模拟正文内容。这是一段没有意义的话,用来模拟正文内容。这是一段没有意义的话,用来模拟正文内容。这是一段没有意义的话,用来模拟正文内容。

\tyear{1919}

\tdoc{这是一部著作,名字有数行之长,令人叹为观止}

\tpart{这是这部名字有数行之长的著作的第一部分}

这是一段没有意义的话,用来模拟正文内容。这是一段没有意义的话,用来模拟正文内容。这是一段没有意义的话,用来模拟正文内容。这是一段没有意义的话,用来模拟正文内容。这是一段没有意义的话,用来模拟正文内容。这是一段没有意义的话,用来模拟正文内容。这是一段没有意义的话,用来模拟正文内容。这是一段没有意义的话,用来模拟正文内容。这是一段没有意义的话,用来模拟正文内容。这是一段没有意义的话,用来模拟正文内容。

\tchapter{这是这部名字有数行之长的著作的第一部分的第一章}

\tvicechapter{(这一章会展示这个模版里大多数的样式。)}

这是一段\underdot{没有意义}\neditor{这是这个模版的强调样式。这种注释是编者添加的,每一页都从 1 开始。}的话,用来模拟\important{正文内容}\neditor{这是这个模版的加粗样式。}。这是\nauthor{这种注释是作者添加的,每一页都接着之前的编号继续。}一段没有意义的话,用来模拟\nend{这种注释是编者添加的,叫做尾注,可能共用于正文中的多处地方。}正文内容。这是一段\italic{没有意义}\neditor{这是这个模版的斜体样式。}的话,用来模拟正文内容。这是一段没有意义的话,用来模拟正文内容。这是一段没有意义的话,用来模拟正文内容。这是一段没有意义的话,用来模拟正文内容。这是一段没有意义的话,用来模拟正文内容。这是一段没有意义的话,用来模拟正文内容。这是一段没有意义的话,用来模拟正文内容。这是一段没有意义的话,用来模拟正文内容。

这是一段没有意义的话,用来模拟正文内容。这是一段没有意义的话,用来模拟正文内容。这是一段没有意义的话,用来模拟正文内容。这是一段没有意义的话,用来模拟正文内容。这是一段没有意义的话,用来模拟正文内容。这是一段没有意义的话,用来模拟正文内容。这是一段没有意义的话,用来模拟正文内容。这是一段没有意义的话,用来模拟正文内容。这是一段没有意义的话,用来模拟正文内容。这是一段没有意义的话,用来模拟正文内容。

这是一段没有意义的话,用来模拟正文内容。这是一段没有意义的话,用来模拟正文内容。这是一段没有意义的话,用来模拟正文内容。这是一段没有意义的话,用来模拟正文内容。这是一段没有意义的话,用来模拟正文内容。这是一段没有意义的话,用来模拟正文内容。这是一段没有意义的话,用来模拟正文内容。这是一段没有意义的话,用来模拟正文内容。这是一段没有意义的话,用来模拟正文内容。这是一段没有意义的话,用来模拟正文内容。

\img[0.6]{tests/sample image.png}[图一 这是一张示意图。图中的人物是托洛茨基]

这是一段没有意义的话,用来模拟正文内容。这是一段没有意义的话,用来模拟正文内容。这是一段没有意义的话,用来模拟正文内容。这是一段没有意义的话,用来模拟正文内容。这是一段没有意义的话,用来模拟正文内容。这是一段没有意义的话,用来模拟正文内容。这是一段没有意义的话,用来模拟正文内容。这是一段没有意义的话,用来模拟正文内容。这是一段没有意义的话,用来模拟正文内容。这是一段没有意义的话,用来模拟正文内容。

\quo{\important{这里是\underdot{一段引用。}}这里是一段引用。这里是一段引用。这里是一段引用。这里是一段引用。这里是一段引用。这里是一段引用。这里是一段引用。这里是一段引用。这里是一段引用。

这是一段没有意义的话,用来模拟正文内容。这是一段没有意义的话,用来模拟正文内容。这是一段没有意义的话,用来模拟正文内容。这是一段没有意义的话,用来模拟正文内容。这是一段没有意义的话,用来模拟正文内容。这是一段没有意义的话,用来模拟正文内容。这是一段没有意义的话,用来模拟正文内容。这是一段没有意义的话,用来模拟正文内容。这是一段没有意义的话,用来模拟正文内容。}

这是一段没有意义的话,用来模拟正文\neditor{在新的一页上,这种注释从 1 开始。}内容。这是一段没有意义的话,用来模拟正文内容。这是一段没有意义的话,用来模拟正文内容。这是一段没有意义的话,用来模拟正文内容。这是一段没有意义的话,用来模拟正文内容。这是一段没有意义的话,用来模拟正文内容。这是一段没有意义的话,用来模拟正文内容。这是一段没有意义的话,用来模拟正文内容。这是一段没有意义的话,用来模拟正文内容。这是一段没有意义的话,用来模拟正文内容。

\hr{}

这是一段没有意义的话,用来模拟正文内容。这是一段没有意义的话,用来模拟正文内容。这是一段没有意义的话,用来模拟正文内容。这是一段没有意义的话,用来模拟正文内容。这是一段没有意义的话,用来模拟正文内容。这是一段没有意义的话,用来模拟正文内容。这是一段没有意义的话,用来模拟正文内容。这是一段没有意义的话,用来模拟正文内容。这是一段没有意义的话,用来模拟正文内容。这是一段没有意义的话,用来模拟正文内容。

\todo{}

这是一段没有意义的话,用来模拟正文内容。\nauthor{在新的一页上,这种注释接着之前的编号继续。}这是一段没有意义的话,用来模拟正文内容。这是一段没有意义的话,用来模拟正文内容。\nend{这是一段没有意义的话,用来模拟正文内容。这是一段没有意义的话,用来模拟正文内容。这是一段没有意义的话,用来模拟正文内容。这是一段没有意义的话,用来模拟正文内容。这是一段没有意义的话,用来模拟正文内容。这是一段没有意义的话,用来模拟正文内容。这是一段没有意义的话,用来模拟正文内容。这是一段没有意义的话,用来模拟正文内容。这是一段没有意义的话,用来模拟正文内容。这是一段没有意义的话,用来模拟正文内容。}这是一段没有意义的话,用来模拟正文内容。这是一段没有意义的话,用来模拟正文内容。这是一段没有意义的话,用来模拟正文内容。这是一段没有意义的话,用来模拟正文内容。这是一段没有意义的话,用来模拟正文内容。这是一段没有意义的话,用来模拟正文内容。这是一段没有意义的话,用来模拟正文内容。

这是一段没有意义的话,用来模拟正文内容。\nauthor{在新的一页上,这种注释接着之前的编号继续。}这是一段没有意义的话,用来模拟正文内容。这是一段没有意义的话,用来模拟正文内容。\nend{这是一段没有意义的话,用来模拟正文内容。这是一段没有意义的话,用来模拟正文内容。这是一段没有意义的话,用来模拟正文内容。这是一段没有意义的话,用来模拟正文内容。这是一段没有意义的话,用来模拟正文内容。这是一段没有意义的话,用来模拟正文内容。这是一段没有意义的话,用来模拟正文内容。这是一段没有意义的话,用来模拟正文内容。这是一段没有意义的话,用来模拟正文内容。这是一段没有意义的话,用来模拟正文内容。}这是一段没有意义的话,用来模拟正文内容。这是一段没有意义的话,用来模拟正文内容。这是一段没有意义的话,用来模拟正文内容。这是一段没有意义的话,用来模拟正文内容。这是一段没有意义的话,用来模拟正文内容。这是一段没有意义的话,用来模拟正文内容。这是一段没有意义的话,用来模拟正文内容。

\tbl[0.8]{tests/sample table.jpg}[表一 这是一张示意表]

这是一段没有意义的话,用来模拟正文内容。这是一段没有意义的话,用来模拟正文内容。这是一段没有意义的话,用来模拟正文内容。这是一段没有意义的话,用来模拟正文内容。这是一段没有意义的话,用来模拟正文内容。这是一段没有意义的话,用来模拟正文内容。这是一段没有意义的话,用来模拟正文内容。这是一段没有意义的话,用来模拟正文内容。这是一段没有意义的话,用来模拟正文内容。这是一段没有意义的话,用来模拟正文内容。

\closing{作者}{1919 年 11 月 1 日}

\info{作者写于 1919 年 11 月 1—2 日之间}{原文是俄语}{载于 1919 年 11 月 2 日《× × ×》第 18 期}{选自《× × ×》第 × 卷 19—28 页}

\tchapter{第一部分的第二章}

这是一段没有意义的话,用来模拟正文内容。这是一段没有意义的话,用来模拟正文内容。这是一段没有意义的话,用来模拟正文内容。这是一段没有意义的话,用来模拟正文内容。这是一段没有意义的话,用来模拟正文内容。这是一段没有意义的话,用来模拟正文内容。这是一段没有意义的话,用来模拟正文内容。这是一段没有意义的话,用来模拟正文内容。这是一段没有意义的话,用来模拟正文内容。这是一段没有意义的话,用来模拟正文内容。

这是一段没有意义的话,用来模拟正文内容。这是一段没有意义的话,用来模拟正文内容。这是一段没有意义的话,用来模拟正文内容。这是一段没有意义的话,用来模拟正文内容。这是一段没有意义的话,用来模拟正文内容。这是一段没有意义的话,用来模拟正文内容。这是一段没有意义的话,用来模拟正文内容。这是一段没有意义的话,用来模拟正文内容。这是一段没有意义的话,用来模拟正文内容。这是一段没有意义的话,用来模拟正文内容。

这是一段没有意义的话,用来模拟正文内容。这是一段没有意义的话,用来模拟正文内容。这是一段没有意义的话,用来模拟正文内容。这是一段没有意义的话,用来模拟正文内容。这是一段没有意义的话,用来模拟正文内容。这是一段没有意义的话,用来模拟正文内容。这是一段没有意义的话,用来模拟正文内容。这是一段没有意义的话,用来模拟正文内容。这是一段没有意义的话,用来模拟正文内容。这是一段没有意义的话,用来模拟正文内容。

\tsection{这是这部名字有数行之长的著作的第一部分的第二章的第一节}

这是一段没有意义的话,用来模拟正文内容。这是一段没有意义的话,用来模拟正文内容。这是一段没有意义的话,用来模拟正文内容。这是一段没有意义的话,用来模拟正文内容。这是一段没有意义的话,用来模拟正文内容。这是一段没有意义的话,用来模拟正文内容。这是一段没有意义的话,用来模拟正文内容。这是一段没有意义的话,用来模拟正文内容。这是一段没有意义的话,用来模拟正文内容。这是一段没有意义的话,用来模拟正文内容。

\tsection{第二章的第二节}

这是一段没有意义的话,用来模拟正文内容。这是一段没有意义的话,用来模拟正文内容。这是一段没有意义的话,用来模拟正文内容。这是一段没有意义的话,用来模拟正文内容。这是一段没有意义的话,用来模拟正文内容。这是一段没有意义的话,用来模拟正文内容。这是一段没有意义的话,用来模拟正文内容。这是一段没有意义的话,用来模拟正文内容。这是一段没有意义的话,用来模拟正文内容。这是一段没有意义的话,用来模拟正文内容。

\tsubsection{这是这部名字有数行之长的著作的第一部分的第二章的第二节的第一小节}

这是一段没有意义的话,用来模拟正文内容。这是一段没有意义的话,用来模拟正文内容。这是一段没有意义的话,用来模拟正文内容。这是一段没有意义的话,用来模拟正文内容。这是一段没有意义的话,用来模拟正文内容。这是一段没有意义的话,用来模拟正文内容。这是一段没有意义的话,用来模拟正文内容。这是一段没有意义的话,用来模拟正文内容。这是一段没有意义的话,用来模拟正文内容。这是一段没有意义的话,用来模拟正文内容。

\tsubsection{第二节的第二小节}

这是一段没有意义的话,用来模拟正文内容。这是一段没有意义的话,用来模拟正文内容。这是一段没有意义的话,用来模拟正文内容。这是一段没有意义的话,用来模拟正文内容。这是一段没有意义的话,用来模拟正文内容。这是一段没有意义的话,用来模拟正文内容。这是一段没有意义的话,用来模拟正文内容。这是一段没有意义的话,用来模拟正文内容。这是一段没有意义的话,用来模拟正文内容。这是一段没有意义的话,用来模拟正文内容。

\tsubsubsection{这是这部名字有数行之长的著作的第一部分的第二章的第二节的第二小节的第一分节}

这是一段没有意义的话,用来模拟正文内容。这是一段没有意义的话,用来模拟正文内容。这是一段没有意义的话,用来模拟正文内容。这是一段没有意义的话,用来模拟正文内容。这是一段没有意义的话,用来模拟正文内容。这是一段没有意义的话,用来模拟正文内容。这是一段没有意义的话,用来模拟正文内容。这是一段没有意义的话,用来模拟正文内容。这是一段没有意义的话,用来模拟正文内容。这是一段没有意义的话,用来模拟正文内容。

\tsubsubsection{第二小节的第二分节}

这是一段没有意义的话,用来模拟正文内容。这是一段没有意义的话,用来模拟正文内容。这是一段没有意义的话,用来模拟正文内容。这是一段没有意义的话,用来模拟正文内容。这是一段没有意义的话,用来模拟正文内容。这是一段没有意义的话,用来模拟正文内容。这是一段没有意义的话,用来模拟正文内容。这是一段没有意义的话,用来模拟正文内容。这是一段没有意义的话,用来模拟正文内容。这是一段没有意义的话,用来模拟正文内容。

\tpart{著作的第二部分}

这是一段没有意义的话,用来模拟正文内容。这是一段没有意义的话,用来模拟正文内容。这是一段没有意义的话,用来模拟正文内容。这是一段没有意义的话,用来模拟正文内容。这是一段没有意义的话,用来模拟正文内容。这是一段没有意义的话,用来模拟正文内容。这是一段没有意义的话,用来模拟正文内容。这是一段没有意义的话,用来模拟正文内容。这是一段没有意义的话,用来模拟正文内容。这是一段没有意义的话,用来模拟正文内容。

\tchapter{第二部分的第一章}

这是一段没有意义的话,用来模拟正文内容。这是一段没有意义的话,用来模拟正文内容。这是一段没有意义的话,用来模拟正文内容。这是一段没有意义的话,用来模拟正文内容。这是一段没有意义的话,用来模拟正文内容。这是一段没有意义的话,用来模拟正文内容。这是一段没有意义的话,用来模拟正文内容。这是一段没有意义的话,用来模拟正文内容。这是一段没有意义的话,用来模拟正文内容。这是一段没有意义的话,用来模拟正文内容。

\tchapter{第二部分的第二章}

这是一段没有意义的话,用来模拟正文内容。这是一段没有意义的话,用来模拟正文内容。这是一段没有意义的话,用来模拟正文内容。这是一段没有意义的话,用来模拟正文内容。这是一段没有意义的话,用来模拟正文内容。这是一段没有意义的话,用来模拟正文内容。这是一段没有意义的话,用来模拟正文内容。这是一段没有意义的话,用来模拟正文内容。这是一段没有意义的话,用来模拟正文内容。这是一段没有意义的话,用来模拟正文内容。

\tyear{1991}

\tdoc{这是另一部著作}

这是一段没有意义的话,用来模拟正文内容。这是一段没有意义的话,用来模拟正文内容。这是一段没有意义的话,用来模拟正文内容。这是一段没有意义的话,用来模拟正文内容。这是一段没有意义的话,用来模拟正文内容。这是一段没有意义的话,用来模拟正文内容。这是一段没有意义的话,用来模拟正文内容。这是一段没有意义的话,用来模拟正文内容。这是一段没有意义的话,用来模拟正文内容。这是一段没有意义的话,用来模拟正文内容。

\tdoc{这是第三部著作}

\tpartnonum{著作的某个部分}

\tchapternonum{有关这个东西的章节}

注意,这些章节是没有编号的。

这是一段没有意义的话,用来模拟正文内容。这是一段没有意义的话,用来模拟正文内容。这是一段没有意义的话,用来模拟正文内容。这是一段没有意义的话,用来模拟正文内容。这是一段没有意义的话,用来模拟正文内容。这是一段没有意义的话,用来模拟正文内容。这是一段没有意义的话,用来模拟正文内容。这是一段没有意义的话,用来模拟正文内容。这是一段没有意义的话,用来模拟正文内容。这是一段没有意义的话,用来模拟正文内容。

\tchapternonum{有关那个东西的章节}

注意,这些章节是没有编号的。

这是一段没有意义的话,用来模拟正文内容。这是一段没有意义的话,用来模拟正文内容。这是一段没有意义的话,用来模拟正文内容。这是一段没有意义的话,用来模拟正文内容。这是一段没有意义的话,用来模拟正文内容。这是一段没有意义的话,用来模拟正文内容。这是一段没有意义的话,用来模拟正文内容。这是一段没有意义的话,用来模拟正文内容。这是一段没有意义的话,用来模拟正文内容。这是一段没有意义的话,用来模拟正文内容。

\tsectionnonum{那个东西的第一节}

这是一段没有意义的话,用来模拟正文内容。这是一段没有意义的话,用来模拟正文内容。这是一段没有意义的话,用来模拟正文内容。这是一段没有意义的话,用来模拟正文内容。这是一段没有意义的话,用来模拟正文内容。这是一段没有意义的话,用来模拟正文内容。这是一段没有意义的话,用来模拟正文内容。这是一段没有意义的话,用来模拟正文内容。这是一段没有意义的话,用来模拟正文内容。这是一段没有意义的话,用来模拟正文内容。

\tsectionnonum{那个东西的第二节}

这是一段没有意义的话,用来模拟正文内容。这是一段没有意义的话,用来模拟正文内容。这是一段没有意义的话,用来模拟正文内容。这是一段没有意义的话,用来模拟正文内容。这是一段没有意义的话,用来模拟正文内容。这是一段没有意义的话,用来模拟正文内容。这是一段没有意义的话,用来模拟正文内容。这是一段没有意义的话,用来模拟正文内容。这是一段没有意义的话,用来模拟正文内容。这是一段没有意义的话,用来模拟正文内容。

\tsubsectionnonum{那个东西的第一小节}

这是一段没有意义的话,用来模拟正文内容。这是一段没有意义的话,用来模拟正文内容。这是一段没有意义的话,用来模拟正文内容。这是一段没有意义的话,用来模拟正文内容。这是一段没有意义的话,用来模拟正文内容。这是一段没有意义的话,用来模拟正文内容。这是一段没有意义的话,用来模拟正文内容。这是一段没有意义的话,用来模拟正文内容。这是一段没有意义的话,用来模拟正文内容。这是一段没有意义的话,用来模拟正文内容。

\tsubsectionnonum{那个东西的第二小节}

这是一段没有意义的话,用来模拟正文内容。这是一段没有意义的话,用来模拟正文内容。这是一段没有意义的话,用来模拟正文内容。这是一段没有意义的话,用来模拟正文内容。这是一段没有意义的话,用来模拟正文内容。这是一段没有意义的话,用来模拟正文内容。这是一段没有意义的话,用来模拟正文内容。这是一段没有意义的话,用来模拟正文内容。这是一段没有意义的话,用来模拟正文内容。这是一段没有意义的话,用来模拟正文内容。

\tsubsubsectionnonum{那个东西的第一分节}

这是一段没有意义的话,用来模拟正文内容。这是一段没有意义的话,用来模拟正文内容。这是一段没有意义的话,用来模拟正文内容。这是一段没有意义的话,用来模拟正文内容。这是一段没有意义的话,用来模拟正文内容。这是一段没有意义的话,用来模拟正文内容。这是一段没有意义的话,用来模拟正文内容。这是一段没有意义的话,用来模拟正文内容。这是一段没有意义的话,用来模拟正文内容。这是一段没有意义的话,用来模拟正文内容。

\tsubsubsectionnonum{那个东西的第二分节}

这是一段没有意义的话,用来模拟正文内容。这是一段没有意义的话,用来模拟正文内容。这是一段没有意义的话,用来模拟正文内容。这是一段没有意义的话,用来模拟正文内容。这是一段没有意义的话,用来模拟正文内容。这是一段没有意义的话,用来模拟正文内容。这是一段没有意义的话,用来模拟正文内容。这是一段没有意义的话,用来模拟正文内容。这是一段没有意义的话,用来模拟正文内容。这是一段没有意义的话,用来模拟正文内容。


% --------------------------------------

\cleardoublepage

\pagestyle{endnotestyle}

\printendnotes

\addcontentsline{toc}{chapter}{注释}

% --------------------------------------

\end{document}

% ------------------------------------------------------------------------------
% EOF
% ------------------------------------------------------------------------------

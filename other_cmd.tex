% important

\newcommand{\important}[1]{%
  \textbf{#1}%
}

% italic

\newcommand{\italic}[1]{%
  \textit{#1}%
}

% underdot

\newcommand{\underdot}[1]{%
  \CJKunderdot{#1}%
}

% hr
% 添加一条水平分割线
% TODO: 在 \quo 前后会产生不同的上下空白

\newcommand{\hr}{%
  \vspace{0.3em}%
  {%
    \centering%
      *     *     *%
    \par%
  }%
}

% quo
% 添加一段引言

\renewenvironment{quote}
{%
  \list{}{\rightmargin=0pt \leftmargin=0pt}%
    \item[]\relax\vspace{0.2em}%
  \fontsize{9.0bp}{9.75bp}\selectfont%
  \hspace*{2em}\parindent=2em\parskip=0pt%
}%
{%
  \vspace{-0.15em}\endlist
}

\newcommand{\quo}[2][]{%
  \begin{quote}%
    #2%
    \ifthenelse{\isempty{#1}}{}{%
      \par\noindent\raggedleft#1%
    }%
  \end{quote}%
}

% closing
% 添加落款信息

\newcommand{\closing}[2]{%
  \vspace{10.5bp}%
  \begin{flushright}%
    \CJKfamily{fs}%
    \fontsize{9.0bp}{9.5bp}\selectfont\ziju{0.010}%
    #1
    \if\relax#2\relax%
    \else%
    \par#2%
    \fi%
  \end{flushright}%
  \vspace{10.5bp}%
}

% info
% 添加作者、出版及编辑等信息

\newcommand{\info}[4]{%
  \noindent
  \begin{minipage}[t]{0.4\textwidth}
    \begin{minipage}[t]{\textwidth}
      \fontsize{9.0bp}{9.5bp}\selectfont\ziju{0.010}#1%
    \end{minipage}

    \rule{0pt}{0.6em}

    \begin{minipage}[t]{\textwidth}
      \fontsize{9.0bp}{9.5bp}\selectfont\ziju{0.010}#2%
    \end{minipage}
  \end{minipage}%
  \hspace{0.2\textwidth}%
  \begin{minipage}[t]{0.4\textwidth}
    \begin{minipage}[t]{\textwidth}
      \fontsize{9.0bp}{9.5bp}\selectfont\ziju{0.010}#3%
    \end{minipage}

    \rule{0pt}{0.6em}

    \begin{minipage}[t]{\textwidth}
      \fontsize{9.0bp}{9.5bp}\selectfont\ziju{0.010}#4%
    \end{minipage}
  \end{minipage}
}

% TODO:为 \img 与 \tbl 添加(可选的):
% 1. 递增的编号
% 2. 随着章节递增的编号

% img
% 在当前位置插入一张图像

\NewDocumentCommand{\img}{O{1} m O{}}{%
  \par\vspace{\dimexpr\baselineskip - 1em\relax}%
  \noindent%
  \makebox[\textwidth][c]{%
    \includegraphics[width=#1\textwidth]{#2}%
  }%
  \IfValueT{#3}{%
    \par\vspace{-0.5em}%
    {\begin{center}
      \CJKfamily{ss}%
      \fontsize{9.0bp}{9.5bp}\selectfont
      \parbox{0.8\dimexpr#1\textwidth\relax}{\centering #3}
    \end{center}}%
  }%
  \par\vspace{\dimexpr\baselineskip - 1.8em\relax}%
}


% tbl
% 在当前位置插入一张表格(标题在上)

\NewDocumentCommand{\tbl}{O{1} m O{}}{%
  \par\vspace{\dimexpr\baselineskip - 1em\relax}%
  \IfValueT{#3}{%
    {\begin{center}
      \CJKfamily{ss}%
      \fontsize{9.0bp}{9.5bp}\selectfont
      \parbox{0.8\dimexpr#1\textwidth\relax}{\centering #3}
    \end{center}}%
    \par\vspace{-0.5em}%
  }%
  \noindent%
  \makebox[\textwidth][c]{%
    \includegraphics[width=#1\textwidth]{#2}%
  }%
  \par\vspace{\dimexpr\baselineskip - 1.8em\relax}%
}

% todo
% 在当前位置插入一个占位框

\newcommand{\todo}{%
  \par\vspace{12bp}%
  \noindent%
  \makebox[\textwidth][c]{%
    \begin{tikzpicture}%
      \node[%
        draw,%
        rectangle,%
        line width=1pt,%
        minimum height=5\baselineskip,%
        text width=0.8\textwidth,%
        align=center%
      ]%
      {这里应该添加一张图像或者一段文字};%
    \end{tikzpicture}%
  }%
  \par\vspace{4bp}%
}
